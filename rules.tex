% !TEX TS-program = pdflatex
% !TEX encoding = UTF-8 Unicode

% This is a simple template for a LaTeX document using the "article" class.
% See "book", "report", "letter" for other types of document.

\documentclass[11pt]{article} % use larger type; default would be 10pt

\usepackage[utf8]{inputenc} % set input encoding (not needed with XeLaTeX)

%%% Examples of Article customizations
% These packages are optional, depending whether you want the features they provide.
% See the LaTeX Companion or other references for full information.

%%% PAGE DIMENSIONS
\usepackage{geometry} % to change the page dimensions
\geometry{a4paper} % or letterpaper (US) or a5paper or....
\geometry{margin=0.8in} % for example, change the margins to 2 inches all round
\geometry{top=1.2in}
% \geometry{landscape} % set up the page for landscape
%   read geometry.pdf for detailed page layout information

\usepackage{graphicx} % support the \includegraphics command and options

% \usepackage[parfill]{parskip} % Activate to begin paragraphs with an empty line rather than an indent

%%% PACKAGES
\usepackage{booktabs} % for much better looking tables
\usepackage{array} % for better arrays (eg matrices) in maths
\usepackage{paralist} % very flexible & customisable lists (eg. enumerate/itemize, etc.)
\usepackage{verbatim} % adds environment for commenting out blocks of text & for better verbatim
\usepackage{subfig} % make it possible to include more than one captioned figure/table in a single float

%\usepackage{lastpage} % Required to determine the last page for the footer
%\usepackage{extramarks} % Required for headers and footers

%%% HEADERS & FOOTERS
\usepackage{fancyhdr} % This should be set AFTER setting up the page geometry
\pagestyle{fancy} % options: empty , plain , fancy
\renewcommand{\headrulewidth}{0pt} % customise the layout...
\lhead{\Tanga \TTthreedots \Tmalta \TTacute  \Ts \Troomen \TTleftcurl \Tlambe \TTacute \Tsilme}
\chead{}
\rhead{\Tcalma \Tlefthook \Tando  \Tanga \TTthreedots \Tmalta \TTacute}
\lfoot{}
\cfoot{\thepage}
\rfoot{}

%++ Font
\renewcommand{\headrulewidth}{1pt}
\renewcommand{\footrulewidth}{0pt}
\usepackage{lmodern}
% \usepackage[utf8]{inputenc}


%%% SECTION TITLE APPEARANCE
\usepackage{sectsty}
\allsectionsfont{\sffamily\mdseries\upshape} % (See the fntguide.pdf for font help)
% (This matches ConTeXt defaults)

%++ More section levels
\usepackage{titlesec}
\setcounter{secnumdepth}{4}

\titleformat{\paragraph}
{\normalfont\normalsize\bfseries}{\theparagraph}{1em}{}
\titlespacing*{\paragraph}
{0pt}{3.25ex plus 1ex minus .2ex}{1.5ex plus .2ex}

%++ List reduced spaces
\usepackage{enumitem}
\setitemize{noitemsep,topsep=3pt,parsep=1pt,partopsep=0pt}
\setenumerate{noitemsep,topsep=3pt,parsep=1pt,partopsep=0pt}

%++ Paragraphs indentation settings
\setlength{\parindent}{0ex}

%++ Tengwar letters
\usepackage[all]{tengwarscript}
\newcommand{\tengfont}{\tengwarformal}

%%% ToC (table of contents) APPEARANCE
\usepackage[nottoc,notlof,notlot]{tocbibind} % Put the bibliography in the ToC
\usepackage[titles,subfigure]{tocloft} % Alter the style of the Table of Contents
\renewcommand{\cftsecfont}{\rmfamily\mdseries\upshape}
\renewcommand{\cftsecpagefont}{\rmfamily\mdseries\upshape} % No bold!
\renewcommand{\contentsname}{Obsah~~\Tcalma \TTrightcurl \Tanto \TTacute \Tanto \Tsilme}

%++ Font color
\usepackage[table]{xcolor}

%++ Albert icon
\usepackage{tikz}
\usepackage{tikzsymbols}
\newcommand{\albert}{\Nursey[][yellow][blue][red]}

\newcommand{\n}{\newline}

%%% END Article customizations

%%% The "real" document content comes below...

\title{
    \tengwarannatarbolditalic \tengmag{4} \Tcalma \Tlefthook \Tando  \Tanga \TTthreedots \Tmalta \TTacute \\
    \textbf{XD--Hra} 
}
\author{
    \textbf{Pravidla hry} \\
    \tengwarannataritalic \tengmag{3.5} \Tanga \TTthreedots \Tmalta \TTacute  \Ts \Troomen \TTleftcurl \Tlambe \TTacute \Tsilme
}
\date{} % Activate to display a given date or no date (if empty),
         % otherwise the current date is printed 

\begin{document}
\maketitle
\tableofcontents

\newpage

\section{Úvod}

\textit{XD--hra} se hraje v obrovských skupinách na pozadí normálního programu.
Její princip je podobný hrám jako jsou Městečko Palermo, AmongUs nebo Werewolf.
Hlavním rozdílem je její délka -- toto není hra na hoďku nebo na jeden večer.
Doba trvání jsou dny, proto \textit{XD}.

Hra se hraje pouze v táboře, každý den od budíčku po večerku.

V pravidlech se občas objeví \albert. To znamená zeptat/říct moderátorovi (Albert).

\section{Hráči a postavy}

\subsection{Cíl hry}

Každý hráč má svůj vlastní cíl v rámci hry, a to jsou:
\begin{itemize}
    \item Nezemřít.
    \item Pokud zemřel, pak být oživen.
    \item Splnit všechny své úkoly, které dostane jako začínající postava.
\end{itemize}

\subsection{Akce}

K tomu, aby hráč dosáhl svého cíle hry, může provádět tyto akce:
\begin{itemize}
    \item Hlasovat pro možného vraha před soudem.
    \item Zavraždit jiného hráče (potřeba mít předmět).
    \item Použít své schopnosti dané vlastní postavou.
    \item Použít nalezený předmět.
\end{itemize}

Podrobnější pravidla pro vykonávání akcí jsou popsána v dalších sekcích.

\subsection{Postavy}

Každý hráč je alespoň jednou postavou ve hře.
Postavou může být jmenovitě postava, nebo jen obyčejný příslušník rasy (např. hobit).
Každá postava a rasa má své speciální schopnosti, frakci a úkoly.
Některé postavy jsou příslušníkem rasy a dědí tak další schopnosti.

\subsection{Vládci}

Vládci jsou hráči, kteří o sobě dohromady od začátku hry vědí a mění některá pravidla hry.
Pokud jsou naživu, od běžných hráčů mohou navíc:
\begin{itemize}
    \item Tvořit, měnit, rušit nebo vyhlašovat \textbf{měnitelná pravidla}.
    \item Jmenovat svého nástupce, pokud vládce zemře. Toto musí (pokud je živý) oznámit \albert pro pozdější ověření.
\end{itemize}

\section{Události}

Každý den nastává mnoho událostí, které mohou spouštět jednotliví hráči prostřednictvím schopností jejich postav nebo použitím nějakého předmětu.  

\subsection{Smrt}

Jestliže chce hráč zavraždit jiného, může to udělat pomocí schopnosti nebo předmětem:
\begin{enumerate}
\item \textbf{Schopností}:
	\begin{enumerate} 
		\item Řekne, co je za postavu (může ukázat kartičku se svou schopností).
        \item Vymyslí způsob smrti adekvátní jeho charakteru (např. Pavouk \& otrávení: malá dírka na zádech,\dots).
        \item Pokud jeho schopnost vyžaduje nějakou speciální akci navíc, musí ji hned provést.
	\end{enumerate}
\item \textbf{Předmětem}:
	\begin{enumerate}
        \item Pokud není uvedeno jinak, hráč může zavraždit jiného hráče \textbf{pouze jedenkrát za hru}, bez ohledu na předmět, který k tomu používá.
        \item Vrah se jako první dotkne oběti.
		\item Vrah ukáže oběti předmět.
		\item Vymyslí způsob smrti související s předmětem.
	\end{enumerate}
\end{enumerate}

Po nalezení mrtvého může nálezce:
\begin{itemize}
	\item \textbf{Oznámit jeho smrt} svému okolí; mrtvý potom řekne, kdy a jak byl zabit; na nástupu bude následovat soud.
	\item \textbf{Oživit mrtvého} (pokud může) tak, že se to ostatní nedozví.
\end{itemize}

Pokud byla smrt oznámena, brzy se objeví obálka na hlasovací tabuli, kde je možné pro ostatní živé hráče vhazovat své hlasy.

Poznámky:
\begin{itemize}
	\item Za nálezce se může vydávat i vrah zavražděného.
	\item Zavražděný zůstane na jeho místě činu do 15 minut nebo do zaznění nástupu, potom to musí co nejdříve oznámit.
	\item Mrtvý \textbf{nesmí prozradit vůbec žádné informace, které se dozvěděl jako živý}, \textbf{\textcolor{red}{nesmí nikomu ani trochu napovědět ani podat žádný náznak o dění ve hře}}, jinak pro něj hra doživotně skončí.
\end{itemize}


\subsection{Soud \& Upálení}

Soud je hlavním procesem tábora za účelem najít pachatele vraždy a toho upálit.
Soudy mají pravidla \textbf{pevná} a \textbf{změnitelná}.
Změnitelná pravidla mohou měnit a vyhlašovat vládci.

\textbf{Pevná pravidla}:
\begin{itemize}
    \item Za každého zavražděného lze odhlasovat a upálit nejvýše jeden živý hráč.
    \item Soud probíhá pouze jednou za den, po večeři.
    \item V táboře je tabule, kde se objeví hlasovací zařízení se jménem každého zemřelého. Lze využít \albert k spravedlivému sčítání hlasů. Hlasovací zařízení nelze nijak sabotovat a zneužívat proti v rosporu s předdefinovanými pravidly.
    \item \albert vyhlásí, kdo získal podle hlasování nejvíce hlasů. Některý vládce pronese závěrečný rozsudek o vině.
	\item Odsouzený nemůže být upálen, pokud má únikovou schopnost -- řekne, co je za postavu, co mu toto poskytuje.
	\item Má-li odsouzený vraždící předmět, může ještě někoho v blízkosti na poslední chvíli zavraždit (dotykem). Zavraždit někoho schopností není možné.
    \item Po pronesení  ,,$X$ je upálen.'' vybraným vládcem hráč zemřel.
\end{itemize}

\textbf{Změnitelná pravidla}. Neexistují. Vládci je musí sestavit a vyhlásit na začátku hry.
Zde jsou standarní doporučení:
{\small
\begin{itemize}
    \item Definovat způsob hlasování.
    \item Hlasování by mělo zajistit, že každý může hlasovat za jednoho mrtvého nejvýše jednou.
    \item Z hlasování by měl s vysokou pravděpodobností vyjít jeden vyník.
    \item Volič by měl mít (demokratickou) možnost výběru, pro koho bude hlasovat.
\end{itemize}
}

\subsection{Oživení}

Mrtvý může být oživen někým jiným a to:
\begin{itemize}
	\item lektvarem nebo
	\item oživovací schopností.
\end{itemize}
Živý nesmí prozradit žádné informace, které se dozvěděl jako mrtvý.

\section{Frakce}
\label{sec:fractions}

Frakcí je skupina hráčů sdílící hlavní herní úkol.

\textbf{Středozemě}.
Běžní obyvatelé Středozemě (tábora), velmi rozmanitá -- žijí tu lidé, elfové, trpaslíci, hobiti, \dots -- a musí si domluvit, jakým způsobem se jejich vláda vypřádá s nadcházející hrozbou.
Ve znaku mají bílý strom.

\textbf{Temná armáda}.
Nepřirozené bytosti sloužící prvnímu generálovi temnot Sauronovi.
Usilují o dobytí celé Středozemě pro sebe.
Ve znaku mají oko Saurona.

\textbf{Neutrální}.
Existují sami za sebe. Nesdílí žádné úkoly s dalšími hráči nebo frakcemi.

\section{Předměty}

Předměty jsou téměř reálně vypadjící předměty, jež lze použít ve hře k akci (např. ochrana, oživení,\dots). Jelikož předměty reálně existují, může vám je někdo vzít, pokud si je nehlídáte (Fyzické násilí je zakázáno.) a předměty můžete kamkoliv schovat.

Předměty můžete používat, pokud je máte fyzicky u sebe.

Pro účinky lektvarů (po vypití) a speciálních předmětů, co to mají napsáno, se ihned zeptejte \albert.

\subsection{Kameny moci}

Tyto předměty nejsou jako jediné označeny popiskem.
Vypadají jako barevné skleněnky.

Jejich použítí je jednorázové.
Efekt kamenu moci je možnost dalšího využití jedné hráčem vybrané schopnosti o jednu "jednotku" více.
Např. to znamená, že pokud hráč může schopnost $A$ použít jednou za hru, s použitým kamenem moci lze schopnost $A$ využít dvakrát za hru.
Navíc, pokud jinou schopnost $B$ lze využít jednou za den, s použitým kamenem moci lze schopnost $B$ využít dvakrát za den, a to i následující den dvakrát.

\end{document}


% \Ttelco \TTthreedots a
% 
% \Tumbar b
% 
% \Tcalma c
% 
% \Tando d
% 
% \Ttelco \TTacute e
% 
% \Tformen f
% 
% \Tanga g
% 
% \Thyarmen h
% 
% \Ttelco \TTdot i
% 
% \TTdot j
% 
% \Tcalma k
% 
% \Tlambe l
% 
% \Tmalta m
% 
% \Tnuumen n
% 
% \Ttelco \TTrightcurl o
% 
% \Tparma p
% 
% \Tquesse q
% 
% \Troomen r
% 
% \Tsilme s
% 
% \Ttinco t
% 
% \Ttelco \TTleftcurl u
% 
% \Tvala v
% 
% \Tvilya w
% 
% \Tcalma \Tlefthook x
% 
% \Tanna \TTtwodotsbelow y
% 
% \Tesse z
% 
% 
% \Ttelco \TTthreedots A
% 
% \Tumbar B
% 
% \Tcalma C
% 
% \Tando D
% 
% \Ttelco \TTacute E
% 
% \Tformen F
% 
% \Tanga G
% 
% \Thyarmen H
% 
% \Ttelco \TTdot I
% 
% \TTdot J
% 
% \Tcalma K
% 
% \Tlambe L
% 
% \Tmalta M
% 
% \Tnuumen N
% 
% \Ttelco \TTrightcurl O
% 
% \Tparma P
% 
% \Tquesse Q
% 
% \Troomen R
% 
% \Tsilme S 
% 
% \Ttinco T
% 
% \Ttelco \TTleftcurl U
% 
% \Tvala V
% 
% \Tvilya W
% 
% \Tcalma \Tlefthook X
% 
% \Tanna \TTtwodotsbelow Y
% 
% \Tesse Z
% 
